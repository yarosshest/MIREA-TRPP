%\graphicspath{{~/Pictures/Screenshots/}} % path to graphics
\graphicspath{{./img/}} % path to graphics

\chapter{Развертывание приложения}
\section{Dockerfile}
Контейнер Docker — это образ Docker, вызванный к жизни.
Это — самодостаточная операционная система, в которой имеется только самое необходимое и код приложения.
Образы Docker являются результатом процесса их сборки, а контейнеры Docker — это выполняющиеся образы.
В самом сердце Docker находятся файлы Dockerfile.
Подобные файлы сообщают Docker о том, как собирать образы, на основе которых создаются контейнеры.
Контейнер бота и базы данных представлен в листингах ниже. \par

\begin{lstlisting}[language=Dockerfile
, caption=\leftline{Dockerfile бота}
, label=lst:DF:bot]
FROM python:latest
WORKDIR /bot
RUN pip install --upgrade pip
ADD requirements.txt requirements.txt
RUN pip install -r requirements.txt
COPY . /bot
CMD ["python", "-u", "main.py"]
\end{lstlisting}

Тут импортируется образ python контейнера , устанавливается requirements и запускается main.py \par

\begin{lstlisting}[language=Dockerfile
, caption=\leftline{Dockerfile базы данных}
, label=lst:DF:db]
FROM postgres:latest
ENV POSTGRES_PASSWORD=postgres
ENV POSTGRES_USER=postgres
ENV POSTGRES_DB=queue
\end{lstlisting}

Тут импортируется образ postgres контейнера , с необходдтимыми параметрами базы данных \par


\section{Docker Compose}
Docker применяется для управления отдельными контейнерами (сервисами), из которых состоит приложение.
Docker Compose используется для одновременного управления несколькими контейнерами, входящими в состав приложения.
Этот инструмент предлагает те же возможности, что и Docker, но позволяет работать с более сложными приложениями.
Ниже представлен листинг Docker Compose проекта \par

\begin{lstlisting}[language=Dockerfile
, caption=\leftline{Dockerfile базы данных}
, label=lst:DС]
version: "3.9"
services:
  bot:
    build:
      context: './bot/'
      dockerfile: "./Dockerfile"
    restart: always
  postgres:
    image: postgres:13.3
    environment:
      POSTGRES_DB: "queue"
      POSTGRES_USER: "postgres"
      POSTGRES_PASSWORD: "postgres"
    build:
      context: './database'
      dockerfile: "./Dockerfile"
    ports:
      - "5432:5432"
\end{lstlisting}

Тут запускаються 2 контейнера bot для бота, postgres для базы данных где выставляеться порт базы данных \par

\section{Развертывание контейнера на скервере}
Docker Compose был развернут на собственном сервере с помощью 2 команд: \par
\begin{verbatim}
  git pull
  docker-compose build
  docker-compose up
\end{verbatim}