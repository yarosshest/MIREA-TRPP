\chapter{Работа с ветвлением и оформление кода}

\section{Форк репозитория}

Чтобы сделать форк, перейдем к нужному репозиторию в GitHub и в верхнем левом
углу нажмем на кнопку "<fork">.
На появившейся странице потребуется
ввести имя форка (Рисунок~\ref{fig:Создание форка в GitHub}).
\img{./gall_3/png_1.png}{Создание форка в GitHub}

\section{Клонирование форка на локальную машину}

Клонирование репозитория производиться командой:
\texttt{git~clone~<адрес~форка>} (Рисунок~\ref{fig:Клонирование форка}).
\img{./gall_3/png_2.png}{Клонирование форка}

\section{Создание двух веток}

Создание новой ветки производиться командой:
\texttt{git~branch~<имя~ветки>} (Рисунок~\ref{fig:Создание двух веток}).
\img{./gall_3/png_3.png}{Создание двух веток}

\section{Коммиты}
Теперь создадим по три коммита в каждой ветке.
Для этого воспользуемся
командой: \texttt{git commit -am "текст коммита"}, которая добавит в индекс
все изменения в индекс и сделате коммит.
Изменения проиллуюстрированы
на рисунках~\ref{fig:Создание трех коммитов в ветке branch1}-\ref{fig:Создание трех коммитов в ветке branch2}.

\img{./gall_3/png_4.png}{Создание трех коммитов в ветке branch1}
\img{./gall_3/png_5.png}{Создание трех коммитов в ветке branch2}

\section{Слияние веток}
\label{3:lb:git:merge:conflict}
Чтобы слить ветку А в ветку Б используется команда: \texttt{git merge Б}.
Если при силянии появился конфликт и git не смог автоматически разрешить его,
то необходимо открыть файл с конфилктом в любом из редакторов и исправть
выделенный участок. Затем останеться выполнить команду \texttt{git commit}
для создания коммита слияния.

Эти действия проиллуюстрирован на рисунке~\ref{fig:Слиянине ветки branch1 в ветку branch2}.
\img{./gall_3/png_6.png}{Слиянине ветки branch1 в ветку branch2}

\section{Отправка всех изменений}
Чтобы, отправить все изменения введем команду
\texttt{git~push~<название~ветки>} для каждой ветки
(Рисунок~\ref{fig:Отправка всех изменений}).
\img{./gall_3/png_7.png}{Отправка всех изменений}

\section{Еще три коммита}
Проведем еще три коммита в ветку branch1 (Рисунок \ref{fig:Создание еще трех коммитов}).
\img{./gall_3/png_8.png}{Создание еще трех коммитов}


\section{Еще одно клонирование репозитория}
Склонируем репозиторий еще раз в другую директорию
(Рисунок~\ref{fig:Новый клон репозитория}).
\img{./gall_3/png_9.png}{Новый клон репозитория}


\section{Добавление коммитов в новый репозиторий}
В новом клоне репозитории сделаем 3 коммита в ветку branch1
(Рисунок~\ref{fig:Три коммита в новом репозитории}).
\img{./gall_3/png_10.png}{Три коммита в новом репозитории}

\section{Отправка изменнеий из нового репозитория}
Выгрузим все изменения из нового в удаленный репозиторий
(Рисунок~\ref{fig:Выгрузка изменений из нового репозитория}).
\img{./gall_3/png_11.png}{Выгрузка изменений из нового репозитория}


\section{Отправка изменнеий из старого репозитория}
Вернемся в старый клон с репозиторием и выгрузим изменения
с опцией \verb|--force| (Рисунок~\ref{fig:Выгрузка изменений из старого репозитория}).
\img{./gall_3/png_12.png}{Выгрузка изменений из старого репозитория}

\section{Разрешение конфликта слияния веток}
Теперь получим все изменения в новом репозитории.
При попытке вытягивания с помощью команды \texttt{git~pull}, git выдал
ошибку из-за конфликта слияния веток и предложил три метода решения:
\begin{verbatim}
	git config pull.rebase false  # merge (the default strategy)
	git config pull.rebase true   # rebase
	git config pull.ff only       # fast-forward only
\end{verbatim}
Введя одну из команд, снова вводим команду вытягивания. Затем устранияем
конфилкт слияния, также как это делалось
выше~(см.~стр.~\pageref{lb:Вытягивание изменений в новый клон репозитория}).
Вывод команд показан на рисунке~\ref{3:fig:git:clone:new:pull}.
\img{./gall_3/png_13.png}{Вытягивание изменений в новый клон репозитория}