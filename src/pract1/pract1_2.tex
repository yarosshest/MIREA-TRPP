\chapter{Системы управления репозиториями}

\section{Создание репозитория на GitHub и на локальной машине}
\subsection{Создание репозитория на GitHub}
Чтобы создать репозиторий в GitHub, нужно:
\begin{enumerate}
    \item Перейти во вкладку "<repositories">.
    \item Нажать на кнопу "<New">.
    \item Ввести название для репозитория и нажать кнопку.
\end{enumerate}
Окно создания репозитория изображено на рисунке \ref{fig:Создание репозитория в GitHub}.
\img{./gall_2/png_1.png}{Создание репозитория в GitHub}

\section{Создание репозитория на локальной машине}
Создание репозитория рассматривалось в прошлой части.
Так что опишем вкратце:
\begin{enumerate}
    \item Создадим каталог, где будет размещаться репозиторий.
    \item Добавим файлы в каталог.
    \item Создадим локальный репозиторий, командой~\texttt{git~init}.
    \item Добавим файлы в индекс, командой~\texttt{git~add~<файлы>}.
    \item Сделаем коммит, чтобы сохранить изменения в репозиторий,
    командой \texttt{git~commit~-m~"текст~коммита"}.
\end{enumerate}
Консольный вывод показан на рисунке \ref{fig:Создание репозитория в Git}.
\img{./gall_2/png_2.png}{Создание репозитория в Git}

\section{Создание SSH-ключа}
Чтобы работать со своего компьютера с GitHub, иметь доступ к проектам,
хранящимся на сервисе, выполнять команды в консоли без постоянного
подтверждения пароля, нужно пройти авторизацию у сервера.
В этом помогают SSH-ключи.

Для создания ключа нужно воспользоваться командой:
\begin{verbatim}
	ssh-keygen -t rsa -b 4096 -C "your_email@example.com"
\end{verbatim}

Во время работы команды потребуется ввести: путь, по которому будет
располагаться созданный ключ, и пароль к создаваемому ключу.
Обычно ключи сохраняются в кталоге .ssh домашней директории. Его содержимое
отображено на рисунке~\ref{fig:Каталог с созданными ssh ключами}.

\img{./gall_2/png_3.png}{Каталог с созданными ssh ключами}

\section{Связывание локального и GitHub репозитория}
Для того, чтобы связать репозиторий на локальной машине и созданным выше
удаленный репозиторием, вначале необходимо зарегестрировать
наш ssh ключ в GitHub. Мы должны его скопировать из консоли
и перейти на страницу для работы с ключами в профиле на GitHub.
Выбираем кнопку “New SSH key”, открывается окно с вводом данных, в поле “key”
вставляем скопированный ключ, в “Title” вводим любое
имя ключа и нажимаем “Add SSH key”.
Добавленный ключ показан на рисунку \ref{fig:Добавленный ssh-ключ в GitHub}

\img{./gall_2/png_4.png}{Добавленный ssh-ключ в GitHub}

Чтобы связать локальный и удаленный репозитории друг с другом необходимо
ввести в консоль следующую команду:
\begin{verbatim}
	git remote add project \
		git@github.com:<ваши имя и название репозитория>.git
	git branch -M main
	git push -u origin main
\end{verbatim}
Вывод команд показан на рисунке~\ref{fig:Связывание репозиториев}.
\img{./gall_2/png_5.png}{Связывание репозиториев}

\section{Создание и слияние новой ветки}
Создание новой ветки в git выполняется
командой: \texttt{git~branch~<имя~ветки>}.

Чтобы изменить текущую ветку есть команда: \texttt{git~checkout~<имя~ветки>}.

Эти две команды можно объединить, чтобы создать новую ветки и сразу на нее переключиться: \texttt{git~checkout~-b~<имя~ветки>}.

Чтобы слить вде ветки, нужно перейти в ту ветку в которую нужно объединить изменения и ввести команду: \texttt{git~merge~<название~ветки>}, как показано
на рисунке \ref{fig:Слияние веток в git}.
\img{./gall_2/png_6.png}{Связывание репозиториев}


\section{Задание варианта}
Выполним цепочку действий в репозитории, согласно 5-ему варианту (27 mod 11):
\begin{itemize}
    \item Клонируем непустой удаленный репозиторий на локальную
    машину (Рисунок \ref{fig:Клонирование удаленного репозиторя}).
    \img{./gall_2/png_7.png}{Клонирование удаленного репозиторя}

    \item Создадим новую ветку и выведим список всех веток(Рисунок \ref{fig:Создание ветки и вывод списка веток}).
    \img{./gall_2/png_8.png}{Создание ветки и вывод списка веток}
    
    \item Произведем 3 коммита в новой ветке (Рисунок \ref{fig:Создание трех коммитов}).
    \img{./gall_2/png_9.png}{Создание трех коммитов}

    \item Выгрузим все изменения в удаленный
    репозиторий (Рисунок \ref{fig:Отправка изменений в удаленный репозиторий}).
    \img{./gall_2/png_10.png}{Отправка изменений в удаленный репозиторий}

    \item Произведем revert предпоследнего коммита
    (Рисунок \ref{fig:revert предпоследнего коммита}).
    \img{./gall_2/png_11.png}{revert предпоследнего коммита}

    \item Выведим в консоли различия между веткой main
    и новой веткой (Рисунок \ref{fig:Различия между ветками}).
    \img{./gall_2/png_12.png}{Различия между ветками}

    \item Сольем новую ветку с master при помощи merge (Рисунок \ref{fig:merge}).
    \img{./gall_2/png_13.png}{merge}
\end{itemize}