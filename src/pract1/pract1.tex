\graphicspath{{/home/arbon/Pictures/Screenshots/}} % path to graphics
\section*{\LARGE{Цель практической работы}}
\addcontentsline{toc}{charter}{Цель практической работы}
Получить навыки по работе с командной строкой и git’ом.
\newpage

\section*{\LARGE{Выполнение практической работы}}
\addcontentsline{toc}{section}{Выполнение практической работы}
\section{Установка и настрока клиент git}
\subsection{Установка git}
Установка в Linux и Unix
\begin{itemize}
	\item Используйте обычный менеджер пакетов вашего дистрибутива.
		Откройте терминал и введите подходящие команды.
	\item Если у вас 21 или более ранняя версия Fedora,
		используйте yum install git.
	\item Для 22 и последующих версий Fedora вводите dnf install git.
	\item Для дистрибутивов, основанных на Debian, например, Ubuntu,
		используйте apt-get: sudo apt-get install git.
\end{itemize}
\begin{figure}[hp]
	\centering
	\includegraphics[width=0.7\textwidth]{Screenshot from 2023-02-14 10-47-05.png}
	\caption{Установка git}
	\label{fig:git:install}
\end{figure}

\subsection{Настройка git}
\begin{enumerate}
	\item Открываем терминал.
	\item Необходимо выполнить следующие команды:
	\begin{verbatim}
		git config --global user.name "Your Name"
		git config --global user.email "your_email@whatever.com"
	\end{verbatim}
	\item Необходимо выполнить следующие команды:
	\begin{verbatim}
		git config --global core.autocrlf input
		git config --global core.safecrlf warn
	\end{verbatim}
	\item Необходимо выполнить следующую команды:
	\begin{verbatim}
		git config --global core.quotepath off
	\end{verbatim}
	\item Необходимо выполнить следующую команды:
	\begin{verbatim}
		git config --global core.quotepath off
	\end{verbatim}
\end{enumerate}
Для проверки верности всех введенных команд, введите:
\begin{verbatim}
	git config --list
\end{verbatim}
\begin{figure}[hp]
	\centering
	\includegraphics[width=0.8\textwidth]{Screenshot from 2023-02-14 11-05-52.png}
	\caption{Настройка git}
	\label{fig:git:config}
\end{figure}

\subsection{Создарние локальный репозиторий и добавление в него несколько файлов}
Выполняем команду \texttt{git init}.
После выполнения данной команды, должно высветиться данное сообщение, показанное на рис. \ref{fig:git:init}.
\begin{figure}[hp]
	\centering
	\includegraphics[width=0.8\textwidth]{Screenshot from 2023-02-14 11-21-16.png}
	\caption{Создание репозитория git}
	\label{fig:git:init}
\end{figure}

\subsection{Добавление файлов в репозиторий}
Далее, командой \texttt{touch file} создадим несколько текстовый файлов.
Чтобы добавить файл в репозиторий необходимо выполнить следующее:
\begin{enumerate}
	\item Вводим команды:
	\begin{verbatim}
		git add <Название вашего файла>
		git commit -m "Ваш текст для коммита"
	\end{verbatim}
\item Чтобы проверить состояние репозитория,
	выполним команду: \texttt{git~status}.
	Команда проверки состояния сообщит, что коммитить нечего.
	Это означает, что в репозитории хранится текущее состояние
	рабочего каталога, и нет никаких изменений, ожидающих записи.
\end{enumerate}
Результат этих действий показан на рисунке \ref{fig:git:first_commit}.
\begin{figure}[hp]
	\centering
	\includegraphics[width=0.8\textwidth]{Screenshot from 2023-02-14 11-32-30.png}
	\caption{Добавление файлов в репозиторий и первый коммит}
	\label{fig:git:first_commit}
\end{figure}
Изменния в репозитории можно отслеживать с помощью команды \texttt{git status}.
\begin{figure}[hp]
	\centering
	\includegraphics[width=0.8\textwidth]{Screenshot from 2023-02-14 11-36-06.png}
	\caption{Отслеживание изменений в репозитории}
	\label{fig:git:status}
\end{figure}
В данной ситуации git распознал, что ваш файл был изменен, но эти изменения не
зафиксированы в репозитории.

\subsection{Индексация изменений}
Для индексации изменений необходимо выполнить команды:

\begin{verbatim}
	git add <Название вашего файла>
	git commit -m "Ваш текст для коммита"
\end{verbatim}

Результат выполнения этих команд показан на рисунке \ref{fig:git:second_commit}.

\begin{figure}[hp]
	\centering
	\includegraphics[width=0.8\textwidth]{Screenshot from 2023-02-14 11-42-12.png}
	\caption{Индексация изменений}
	\label{fig:git:second_commit}
\end{figure}
Теперь Git знает об изменении.

\subsection{Индексация и коммит}
Создадим в директории еще 3 файла
\begin{figure}[hp]
	\centering
	\includegraphics[width=0.8\textwidth]{Screenshot from 2023-02-14 11-47-06.png}
	\caption{Создание файлов}
	\label{fig:git:files}
\end{figure}
Отредактирум эти файл. Теперь  закуммитим все изменения, при этом чтобы изменения в a.html и b.html были одним
коммитом, в то время как изменения в c.html логически не связаны с первыми двумя
файлами и должны идти отдельным коммитом.
Введем следующие команды:
\begin{verbatim}
	git add a.html b.html
	git commit -m "Изменения в a.html и b.html"
	git add c.html
	git commit -m "Изменения в c.html"
\end{verbatim}

\subsection{Коммиты нескольких изменений}
Чтобы добавить несколько изменений в один коммит, необходимо выполнить следующие команды:
\begin{verbatim}
	git add .
	git commit -m "Ваш текст для коммита"
\end{verbatim}
Результат выполнения этих команд показан на рисунке \ref{fig:git:commit}.
\begin{figure}[hp]
	\centering
	\includegraphics[width=0.8\textwidth]{Screenshot from 2023-02-14 11-57-00.png}
	\caption{Коммиты нескольких изменений}
	\label{fig:git:commit}
\end{figure}

\subsection{Просмотр истории коммитов}
Для просмотра истории коммитов необходимо выполнить команду \texttt{git log}.
Результат выполнения этой команды показан на рисунке \ref{fig:git:log}.
\begin{figure}[hp]
	\centering
	\includegraphics[width=0.8\textwidth]{Screenshot from 2023-02-14 12-00-00.png}
	\caption{Просмотр истории коммитов}
	\label{fig:git:log}
\end{figure}
Команда \texttt{git log} позволяет контролировать формат выводимой информации:
\begin{verbatim}
	git log --pretty=oneline
\end{verbatim}
Результат выполнения этой команды показан на рисунке \ref{fig:git:log:pretty}.
\begin{figure}[hp]
	\centering
	\includegraphics[width=0.8\textwidth]{Screenshot from 2023-02-14 12-02-00.png}
	\caption{Просмотр истории коммитов с помощью команды \texttt{git log --pretty=oneline}}
	\label{fig:git:log:pretty}
\end{figure}
Для контроля отображения записей можно использовать следующие параметры:
\begin{itemize}
	\item \texttt{--since} - отображать записи, созданные после указанной даты
	\item \texttt{--until} - отображать записи, созданные до указанной даты
	\item \texttt{--author} - отображать записи, созданные указанным автором
	\item \texttt{--grep} - отображать записи, содержащие указанную строку
\end{itemize}
Для примера выполним команду
\begin{verbatim}
	git log --pretty=format:"%h %ad | %s%d [%an]" --graph --date=short
\end{verbatim}
Рассмотрим её в деталях:
\begin{itemize}
	\item \texttt{\%h} - сокращенный хэш коммита
	\item \texttt{\%d} - дополнения коммита
	\item \texttt{\%ad} - дата коммита
	\item \texttt{\%s} - сообщение коммита
	\item \texttt{\%an} - автор коммита
	\item \texttt{--graph} - отображение графа коммитов
	\item \texttt{--date=short} - короткая дата
\end{itemize}
Результат выполнения этой команды показан на рисунке \ref{fig:git:log:pretty:format}.
\begin{figure}[hp]
	\centering
	\includegraphics[width=0.8\textwidth]{Screenshot from 2023-02-14 12-06-00.png}
	\caption{Просмотр истории коммитов с помощью команды \texttt{git log --pretty=format:"\%h \%ad | \%s\%d [\%an]" --graph --date=short}}
	\label{fig:git:log:pretty:format}
\end{figure}

\subsection{Получение старых версий}
Для получение старой версии нашего каталога необходимо узнать для начала хэши,для этого воспользуемся командой из предыдущего шага:
\begin{verbatim}
	git log --pretty=format:"%h %ad | %s%d [%an]" --graph --date=short
\end{verbatim}
Результат выполнения этой команды показан на рисунке \ref{fig:git:log:pretty:format}.
\begin{figure}[hp]
	\centering
	\includegraphics[width=0.8\textwidth]{Screenshot from 2023-02-14 12-06-00.png}
	\caption{Просмотр истории коммитов с помощью команды \texttt{git log --pretty=format:"\%h \%ad | \%s\%d [\%an]" --graph --date=short}}
	\label{fig:git:log:pretty:format}
\end{figure}
Теперь, зная хэши, можно получить старые версии файлов, для этого необходимо выполнить команду:
\texttt{git checkout хэш\_файла}
Результатом будет:
\begin{figure}
	\centering
	\includegraphics[width=0.8\textwidth]{Screenshot from 2023-02-14 12-09-00.png}
	\caption{Получение старых версий}
	\label{fig:git:checkout}
\end{figure}

\subsection{Отмена изменений (до индексации)}
Для отмены изменений необходимо выполнить команду:
\begin{verbatim}
	git checkout master
\end{verbatim}
Результат выполнения этой команды показан на рисунке \ref{fig:git:checkout:--}.
\begin{figure}[hp]
	\centering
	\includegraphics[width=0.8\textwidth]{Screenshot from 2023-02-14 12-11-00.png}
	\caption{Отмена изменений (до индексации)}
	\label{fig:git:checkout:--}
\end{figure}

\subsection{Отмена изменений (после индексации)}
Для отмены изменений необходимо выполнить команду:
\begin{verbatim}
	git reset HEAD имя_файла
\end{verbatim}
Результат выполнения этой команды показан на рисунке \ref{fig:git:reset:HEAD}.
\begin{figure}[hp]
	\centering
	\includegraphics[width=0.8\textwidth]{Screenshot from 2023-02-14 12-13-00.png}
	\caption{Отмена изменений (после индексации)}
	\label{fig:git:reset:HEAD}
\end{figure}
После этого нужно убрать нежелательный комментарий.
\begin{verbatim}
	git checkout -- project.html
\end{verbatim}

\subsection{Отмена коммита}
Для отмены коммита необходимо выполнить команду:
\begin{verbatim}
	git revert --hard HEAD~1
\end{verbatim}
После выполнения данной команды откроется редактор, который предложит вам
отредактировать коммит или вы можете оставить его как есть. Можно также
воспользоваться параметром \texttt{--no-edit}
Результатом данной команды будет:
\begin{figure}[hp]
	\centering
	\includegraphics[width=0.8\textwidth]{Screenshot from 2023-02-14 12-16-00.png}
	\caption{Отмена коммита}
	\label{fig:git:revert:--hard:HEAD~1}
\end{figure}
Проверим историю коммитов при помощи команды
\begin{verbatim}
	git log --pretty=format:"%h %ad | %s%d [%an]" --graph --date=short
\end{verbatim}
Результат выполнения этой команды показан на рисунке \ref{fig:git:log:pretty:format}.
\begin{figure}[hp]
	\centering
	\includegraphics[width=0.8\textwidth]{Screenshot from 2023-02-14 12-06-00.png}
	\caption{Просмотр истории коммитов с помощью команды \texttt{git log --pretty=format:"\%h \%ad | \%s\%d [\%an]" --graph --date=short}}
	\label{fig:git:log:pretty:format}
\end{figure}

\section{Управление репозиториями}
\subsection{Создание аккаунта на GitHub}
Для создания аккаунта на GitHub необходимо зайти на сайт \url{https://github.com} и нажать на кнопку
\texttt{Sign up for GitHub}. Далле идет регистрация, после чего необходимо подтвердить свой аккаунт по почте.

\subsection{Создание ssh-ключа}
Сначала проверим, есть ли уже на компьютере ключ. По умолчанию SSH-ключи
хранятся в каталоге \texttt{~/.ssh}, поэтому нужно проверить содержимое этого каталога.
Открываем консоль и вводим команду:
\begin{verbatim}
	ssh-keygen -t rsa -b 4096 -C "your_mail@example.com"
\end{verbatim}
Далее нужно указать расположение файла для сохранения ключа. Если вы не
введёте путь до файла и просто нажмёте Enter, ключ сохранится в файле,
указанном в скобках.
Теперь нужно установить пароль к вашему ключу и дважды ввести его. Если вы не
хотите вводить пароль каждый раз, когда используете ключ, пропустите этот шаг,
нажав «Enter», и ничего не вводите.
